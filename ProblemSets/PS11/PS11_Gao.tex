\documentclass[12pt,english]{article}
\usepackage{mathptmx}

\usepackage{color}
\usepackage[dvipsnames]{xcolor}
\definecolor{darkblue}{RGB}{0.,0.,139.}

\usepackage[top=1in, bottom=1in, left=1in, right=1in]{geometry}

\usepackage{amsmath}
\usepackage{amstext}
\usepackage{amssymb}
\usepackage{setspace}
\usepackage{lipsum}
\usepackage{caption}
\usepackage[authoryear]{natbib}
\usepackage{url}
\usepackage{booktabs}
\usepackage[flushleft]{threeparttable}
\usepackage{graphicx}
\usepackage[english]{babel}
\usepackage{pdflscape}
\usepackage{float} 
\usepackage{subfigure}
\usepackage[unicode=true,pdfusetitle,
 bookmarks=true,bookmarksnumbered=false,bookmarksopen=false,
 breaklinks=true,pdfborder={0 0 0},backref=false,
 colorlinks,citecolor=black,filecolor=black,
 linkcolor=black,urlcolor=black]
 {hyperref}
\usepackage[all]{hypcap} % Links point to top of image, builds on hyperref
\usepackage{breakurl}    % Allows urls to wrap, including hyperref

\linespread{2}

\begin{document}



\begin{singlespace}
\title{Titanic Survival Prediction by Machine Learning}
\end{singlespace}

\author{Ningjing Gao\\University of oklahoma}
% \date{\today}
\date{May 4, 2020}

\maketitle

\begin{abstract}

	The sinking of Titanic caused thousands of passengers died.  The lack of lifeboats and the mistakes in rescue measures are the main reasons for the death of so many passengers in this disaster. The data collected from https://www.kaggle.com/c/titanic/data shows the features like the gender, the age even the ticket class level decides whether they can survive. This paper will base on the passengers feature to build machine learning models and Naive Bayes Classifiers to analysis. This can predict who has more possible to survive. Then use the model to predict whether or not the passenger survived the sinking of the Titanic.

\end{abstract}
\section{introduction}
Through the movie Titanic, the "Titanic" disaster on April 15, 1912 is known all over the world. As in the movie, a small number of lifeboats can only save a small number of passengers. First-class passengers, women and children are given priority. This article will make predictions about which passengers will survive the sinking through machine learning. This algorithm can predict different combinations of survival functions. The data analysis will be completed after the application of the algorithm and the accuracy will be checked. Different algorithms are compared based on accuracy and best performance. It is recommended to use the model for prediction.

\section{Research Finding }


Depending on the bar chart, we can learn the probability of 3rd class for people not survived is larger than 1st and  2nd class, and the 1st class for survived is higher than others. Hence, the relationship between survived and Pclass is dependent


\begin{figure}[H]
\centering
\includegraphics[scale=0.6]{ggplot.png}
\caption{The Survival Probability of The Different Ticket Class}
\label{Fig.The Survival Probability of The Different Ticket Class}
\end{figure}


%----------------------------------------
% Table 1
%----------------------------------------
\begin{table}[ht]
\caption{Survival Predictions for Each Combination of Pclass,Sex Values}
\label{tab:descriptives} 
\centering
\begin{threeparttable}
\begin{tabular}{lcccc}
&&&&\\
\multicolumn{5}{l}{\emph{Panel : Sex = female}}\\

\toprule
                                                        &Pclass1 & Pclass 2 & Pclass3\\ 
\midrule
Survived0                                               &0.003367003& 0.006734007& 0.080808081\\
Survived1                                               &0.102132435& 0.078563412&0.080808081\\
&&&&\\



\multicolumn{5}{l}{\emph{Panel : Sex = male}}\\

\toprule
                                                         &Pclass1 & Pclass 2 & Pclass3\\ 
\midrule
Survived0                                                &0.086419753& 0.102132435& 0.336700337\\
Survived1                                                & 0.050505051& 0.019079686 &0.0527497191\\
&&&&\\


\multicolumn{5}{l}{\emph{Panel : Sex = female}}\\

\toprule
                                                         &Pclass1 & Pclass 2 & Pclass3&sum\\ 
\midrule
Survived0                                                &0.003367003&0.006734007& 0.080808081&0.090909091\\
Survived1                                                &0.102132435&0.078563412&0.08080808
1& 0.261503928\\
sum                                                      &0.105499439&0.085297419& 0.161616162& 0.352413019
&&&&\\
\multicolumn{5}{l}{\emph{Panel : Sex = male}}\\
\toprule
                                                         &Pclass1 & Pclass 2 & Pclass3&sum\\ 
\midrule
Survived0                                                &0.086419753& 0.102132435& 0.336700337& 0.525252525\\
Survived1                                                &0.050505051& 0.019079686& 0.052749719& 0.122334456\\
sum                                                      & 0.136924804 &0.121212121& 0.389450056& 0.647586981\\
&&&&\\
\multicolumn{5}{l}{\emph{Panel : Sex = sum}}\\
\toprule
                                                         &Pclass1 & Pclass 2 & Pclass3&sum\\ 
\midrule
Survived0                                                &0.089786756& 0.108866442& 0.417508418& 0.616161616\\
Survived1                                                &0.152637486& 0.097643098& 0.133557800& 0.383838384\\
sum                                                      &0.242424242& 0.206509540& 0.551066218& 1.000000000\\     

\bottomrule
\end{tabular}
\footnotesize Notes: The prediction using these predictors, Pr[Survival|Pclass, Sex], is simply a function of Pclass,Sex.

\end{threeparttable}
\end{table}












%----------------------------------------
% Table 
%----------------------------------------
\begin{table}[ht]
\caption{Survival Predictions for Each Combination of Pclass}
\label{tab:descriptives} 
\centering
\begin{threeparttable}
\begin{tabular}{lcccc}
&&&&\\

\toprule
                                                        &Pclass1 & Pclass 2 & Pclass3\\ 
\midrule
Survived0                                               &0.08978676& 0.10886644& 0.41750842\\
Survived1                                               &0.15263749& 0.09764310&0.13355780\\
&&&&\\
\bottomrule
\end{tabular}
\footnotesize Notes: survived0 is not survived, survived1 is survived.

\end{threeparttable}
\end{table}







%----------------------------------------
% Table 
%----------------------------------------
\begin{table}[ht]
\caption{marginal distributions for both survival status and passenger class
}
\label{tab:descriptives} 
\centering
\begin{threeparttable}
\begin{tabular}{lcccc}
&&&&\\

\toprule
                                                        &Pclass1 & Pclass 2 & Pclass3&sum\\ 
\midrule
Survived0                                               & 0.08978676& 0.10886644& 0.41750842& 0.61616162\\
Survived1                                               &0.15263749& 0.09764310& 0.13355780& 0.38383838\\                      
sum                                                     &0.24242424& 0.20650954& 0.55106622& 1.00000000
&&&&\\
\bottomrule
\end{tabular}
\footnotesize Notes: survived0 is not survived, survived1 is survived.
\end{threeparttable}
\end{table}


\begin{figure}[H]
\centering
\includegraphics[scale=0.6]{p-value.png}
\caption{visualize the relationship between survival status and passenger class}
\label{Fig.visualize the relationship between survival status and passenger class}
\end{figure}



\bibliographystyle{plain}
\bibliography{references}



\end{document}