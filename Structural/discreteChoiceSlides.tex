%==========================================================
% Preamble
%==========================================================
% Fonts/languages
\documentclass[english,aspectratio=169,12pt,xcolor=dvipsnames]{beamer}
%\usepackage[orientation=landscape,size=custom,width=16,height=9,scale=0.45,debug]{beamerposter} % change to 16x9 aspect ratio from default 4x3
\usepackage[T1]{fontenc}
\usepackage[latin9]{inputenc}
\usepackage{babel}            % For bibliographies

\usepackage[scaled]{beramono}  % Load typewriter font first, since Arial doesn't have one
\usepackage{helvet}            % For Arial font
\renewcommand{\familydefault}{\sfdefault}
%\renewcommand{\ttfamily}{beramono}

%\usepackage{mathpazo}         % For Palatino font
%%\usepackage{mathptmx}         % For Times New Roman font
%\usefonttheme{serif}          % To have a serif font (instead of Beamer's default sans serif)

% Colors: see  http://www.math.umbc.edu/~rouben/beamer/quickstart-Z-H-25.html
\usepackage{color}
\usepackage[dvipsnames]{xcolor}
\definecolor{dgreen}{rgb}{0.,0.6,0.}
\definecolor{forest}{RGB}{34.,139.,34.}
\definecolor{byublue}{RGB}{0.,30.,76.}
\definecolor{dukeblue}{RGB}{0.,0.,156.}
\definecolor{oucrimson}{RGB}{132.,22.,23.}
\definecolor{whitey}{RGB}{255.,255.,255.}

% Beamer Themes
%\usetheme{Ilmenau}
%%%%%%%%%%%%%%%%%%%%%%%%%%%%%%%%%%%%%%%%%%%%%%%%
% NOTE: With Ilmenau style, to get the bullets %
% looking right, do one section and one sub-   %
% section for each set of bullets              %
% This is not necessary with Frankfurt         %
%%%%%%%%%%%%%%%%%%%%%%%%%%%%%%%%%%%%%%%%%%%%%%%%

\usetheme{CapeTown}
\setbeamercolor{section in head/foot}{bg=oucrimson,fg=whitey}  % use this to adjust the coloring of CapeTown
\setbeamercolor{frametitle}{bg=whitey,fg=oucrimson}            % use this to adjust the coloring of CapeTown
\setbeamercolor{alerted text}{fg=oucrimson}
\usecolortheme[named=oucrimson]{structure}                   % change the color theme
%\usecolortheme[named=RawSienna]{structure}                  % change the color theme
%\usecolortheme[named=byublue]{structure}                    % change the color theme
%\setbeamertemplate{itemize items}[default]                  % change bullet style. options are: default (triangle), square, circle, ball
\setbeamercovered{invisible}

%\usetheme{Frankfurt}
%\setbeamercolor{section in head/foot}{fg=whitey,bg=byublue}  % use this to adjust the coloring of Frankfurt
%\usecolortheme[named=dukeblue]{structure}                   % change the color theme
%%\usecolortheme[named=RawSienna]{structure}                 % change the color theme
%%\usecolortheme[named=byublue]{structure}                   % change the color theme
%\setbeamertemplate{navigation symbols}{}                    % remove annoying navigation symbols
%%\setbeamertemplate{headline}{}                             % remove header line
%\setbeamertemplate{footline}{}                              % remove footer line
%\setbeamercovered{invisible}                                % transparent shows bullets in gray; invisible hides bullets
%
%\usepackage{remreset}                                       % For FRANKFURT (and ILMENAU) style
%\makeatletter                                               % to allow circles
%\@removefromreset{subsection}{section}                      % to appear by frame
%\makeatother                                                % instead of by
%\setcounter{subsection}{1}                                  % subsection.

% Layout
%\usepackage[bottom]{footmisc}                         % Forces footnotes on bottom

% Useful Packages
%\usepackage{bookmark}                                              % For speedier bookmarks
\usepackage{booktabs}                                              % For more professional tables
\usepackage{babel}                                                 % For multilingual output
\usepackage{amsthm}                                                % For detailed theorems
\usepackage{amssymb}                                               % For fancy math symbols
\usepackage{amsmath}                                               % For awesome equations/equation arrays
\usepackage{float}                                                 % For improved float manipulation
\usepackage{prettyref}                                             % For pretty references
\usepackage{array}                                                 % For tubular tables
\usepackage{longtable}                                             % For long tables
\usepackage[flushleft]{threeparttable}                             % For three-part tables
\usepackage{multicol}                                              % For multi-column cells
\usepackage{graphicx}                                              % For shiny pictures
\usepackage{subfig}                                                % For sub-shiny pictures
\usepackage{enumerate}                                             % For cusomtizable lists
\usepackage{listings}                                              % For ``verbatim'' code
\lstset{basicstyle=\ttfamily\footnotesize,breaklines=true}
%\usepackage{outline}                                               % For an outline environment (similar to enumerate, but can go six levels deep)
\usepackage{pdflscape}                                             % For landscape-oriented pages
%\usepackage{pstricks,pst-node,pst-tree,moredefs}                   % For trees
%\usepackage{tikz,pgfplots}                                         % For tikz figures
%\pgfplotsset{compat=1.7}
%\pgfplotsset{scaled y ticks=base 10:2}

% Custom operators for variance, covariance, and correlation
\newcommand{\Var}{\operatorname{Var}}
\newcommand{\Cov}{\operatorname{Cov}}
\newcommand{\Corr}{\operatorname{Corr}}

% Bib
\usepackage[authoryear]{natbib}                                    % Bibliography
\renewcommand{\bibsection}{\subsubsection*{\bibname } }            % Allows Bibliographies in Beamer
\usepackage{url}                                                   % Allows urls in bib

% Links
\usepackage{hyperref}                                              % Always add hyperref (almost) last
%\hypersetup{unicode=true,bookmarksnumbered=true,bookmarksopen=true,bookmarksopenlevel=4,
% breaklinks=true,pdfborder={0 0 0},colorlinks=false,citecolor=black,filecolor=black,linkcolor=black,urlcolor=black}
\hypersetup{unicode=true,breaklinks=true,pdfborder={0 0 0},colorlinks=true,urlcolor=oucrimson,linkcolor=oucrimson}

% Julia definition (c) 2014 Jubobs
%\lstdefinelanguage{Julia}%
%  {morekeywords={abstract,begin,break,case,catch,const,continue,do,else,elseif,%
%      end,export,false,for,function,immutable,import,importall,if,in,%
%      macro,module,otherwise,quote,return,switch,true,try,type,typealias,%
%      using,while},%
%   sensitive=true,%
%   alsoother={$},%
%   morecomment=[l]\#,%
%   morecomment=[n]{\#=}{=\#},%
%   morestring=[s]{"}{"},%
%   morestring=[m]{'}{'},%
%}[keywords,comments,strings]%

%\lstset{%
%    language         = Julia,
%    basicstyle       = \ttfamily,
%    keywordstyle     = \bfseries\color{blue},
%    stringstyle      = \color{magenta},
%    commentstyle     = \color{ForestGreen},
%    showstringspaces = false,
%}

%%==========================================================
%%Change bullet spacing Chetty-style:
%\makeatletter
%\newcommand{\setlistspacing}[2]{\def\@ld{#1}\expandafter\def\csname
%@list\romannumeral\@ld \endcsname{\leftmargin\csname
%leftmargin\romannumeral\@ld \endcsname
              %\topsep    #2
              %\parsep    0\p@   \@plus\p@
              %\itemsep   #2}}
%\makeatother
%\setlistspacing{1}{2.5ex}
%%==========================================================

%==========================================================
% How to cite an author using BibTeX:
% \citet{hotz_et_al2002} --> Hotz et al. (2002)
% \citeauthor{hotz_et_al2002} ---> Hotz et al.
% \citep{hotz_et_al2002} --> (Hotz et al., 2002)
% \citealp{hotz_et_al2002} --> Hotz et al., 2002
% \citealt{hotz_et_al2002} --> Hotz et al. 2002
% \citeyear{hotz_et_al2002} --> 2002
% \citeyearpar{hotz_et_al2002} --> (2002)
% \nocite{hotz_et_al2002} --> [only shows up in bibliography]
%
% If you add a "*" to the command, it will list all authors:
% \citet*{hotz_et_al2002} --> Hotz, Xu, Tienda, and Ahituv (2002)
% \citeauthor*{hotz_et_al2002} ---> Hotz, Xu, Tienda, and Ahituv
% \vdots
% etc.
%==========================================================

%==========================================================
% How to use overlay specifications in Beamer:
%
% Overlay specifications are given in pointed brackets (<,>) and
% indicate which slide the corresponding information should appear
% on.
% The specification <1-> means ``display from slide 1 on.'' <1-3>
% means ``display from slide 1 to slide 3.'' <-3,5-6,8-> means
% ``display on all slides except slides 4 and 7.''
% Here is an example:
% \begin{itemize}
% \item<1> $abcadcabca$
% \item<1-2> $abcabcabca$
% \item<1-2> $accaccacca$
% \item<1> $bacabacaba$
% \item<1,3> $cacdaccacc$
% \item<1-2> $caccaccacc$
% \end{itemize}
%==========================================================

%==========================================================
% How to add ToC before each section:
\AtBeginSection[]{
 \frame<beamer>{ 
   \frametitle{Outline}   
   \tableofcontents[currentsection] 
}
}
%==========================================================

\title{Structural Models of Utility Maximization}
\author{Tyler Ransom}
\institute[OU Econ]{\normalsize{University of Oklahoma, Dept. of Economics}}
\date{April 16, 2020}

\begin{document}

{
\setbeamertemplate{footline}{} 
\frame[noframenumbering]{\titlepage}
}

\AtBeginSection[] {
\begin{frame}
    \frametitle{Outline}
    \tableofcontents[currentsection]
\end{frame}
}



\section{Intro}
\begin{frame}{Today's plan}
\begin{enumerate}
\item Describe static discrete choice models
\item How do they fit in with other data science models we've talked about in this class?
\item Derive logit/probit probabilities from intermediate microeconomic theory
\item Go through examples of how to estimate
\item How discrete choice models relate to sample selection bias
\end{enumerate}

\bigskip{}
\bigskip{}

\footnotesize{Note: These slides are based on the introductory lecture of a PhD course taught at Duke University by Peter Arcidiacono, and are used with permission. That course is based on Kenneth Train's book \emph{Discrete Choice Methods with Simulation}, which is freely available \href{https://eml.berkeley.edu/books/train1201.pdf}{here} (PDF).}
\end{frame}



\section{Discrete choice}
\begin{frame}{What are discrete choice models?}
\begin{itemize}
\item Discrete choice models are one of the workhorses of structural economics
\item Deeply tied to economic theory:
    \begin{itemize}
    \item utility maximization
    \item revealed preference
    \end{itemize}
\item Used to model ``utility'' (broadly defined), for example:
    \begin{itemize}
    \item consumer product purchase decisions
    \item firm market entry decisions
    \item investment decisions
    \end{itemize}
\end{itemize}
\end{frame}



\begin{frame}{Why use discrete choice models?}
\begin{itemize}
\item Provides link between human optimization behavior and economic theory
\item Parameters of these models map directly to economic theory
\item Parameter values can quantify a particular policy
\item Can be used to form counterfactual predictions (e.g. by adjusting certain parameter values)
\item Allows a research to quantify ``tastes''
\end{itemize}
\end{frame}



\begin{frame}{Why \textbf{not} use discrete choice models?}
\begin{itemize}
\item They're not the best predictive models
    \begin{itemize}
    \item Trade-off between out-of-sample prediction and counterfactual prediction
    \end{itemize}
\item You don't want to form counterfactual predictions, you just want to be able to predict handwritten digits
\item You aren't interested in economic theory
\item The math really scares you
\item You don't like making assumptions
    \begin{itemize}
    \item e.g. that decision-makers are rational
    \end{itemize}
\end{itemize}
\end{frame}



\begin{frame}{Example of a discrete choice model}
\begin{itemize}
\item Cities in the Bay Area are interested in how the introduction of rideshare services will impact ridership on Bay Area Rapid Transit (BART)
\item Questions that cities need to know the answers to:
    \begin{itemize}
    \item Is rideshare a substitute for public transit or a complement?
    \item How inelastic is demand for BART? Should fares be $\uparrow$ or $\downarrow$?
    \item Should BART services be scaled up to compete with rideshares?
    \item Will the influx of rideshare vehicles increase traffic congestion / pollution?
    \end{itemize}
\item Each of these questions requires making a counterfactual prediction
\item In particular, need a way to make such a prediction confidently and in a way that is easy to understand
\end{itemize}
\end{frame}



\begin{frame}{Properties of discrete choice models}
\begin{enumerate}
\item Agents choose from among a \textbf{finite} set of alternatives (called the \emph{choice set})
\item Alternatives in choice set are \textbf{mutually exclusive}
\item Choice set is \textbf{exhaustive}
\end{enumerate}
\end{frame}



\begin{frame}{Example illustrating these properties}
\begin{itemize}
\item In San Francisco, people can commute to work by the following (and \emph{only} the following) methods:
    \begin{itemize}
    \item Drive a personal vehicle (incl. motorcycle)
    \item Carpool in a personal vehicle
    \item Use taxi/rideshare service (incl. Uber, Lyft, UberPool, LyftLine, etc.)
    \item BART (bus, train, or both)
    \item Bicycle
    \item Walk
    \end{itemize}
\end{itemize}
\end{frame}



\section{Math}
\begin{frame}{Mathematically representing utility}
Let $d_i$ indicate the choice individual (or decision-maker) $i$ makes where $d_i\in\{1,\cdots, J\}$.
Individuals choose $d$ to maximize their utility, $U$.  $U$ generally is written as:
\begin{equation}
U_{ij}=u_{ij}+\varepsilon_{ij}
\end{equation}
where:
\begin{enumerate}
\item  $u_{ij}$ relates observed factors to the utility individual $i$ receives from choosing option $j$
\item $\varepsilon_{ij}$ are unobserved to the researcher but observed to the individual
\item $d_{ij}=1$ if $u_{ij}+\varepsilon_{ij}>u_{ij'}+\varepsilon_{ij'}$ for all $j'\neq j$
\end{enumerate}
\end{frame}



\begin{frame}{Breakdown of the assumptions}
\begin{itemize}
\item Examples of what's in $\varepsilon$
    \begin{itemize}
    \item Person's mental state when making the decision
    \item Choices of friends or relatives (maybe, depends on the data)
    \item $\vdots$
    \item Anything else about the person that is not in our data
    \end{itemize}
\item Reasonable to assume additive separability?
    \begin{itemize}
    \item This is a big assumption: that there are no interactive effects between unobservable and observable factors
    \item This results in linear separation regions and may be too restrictive
    \item For now, go with it, and remember that there are no free lunches
    \end{itemize}
\end{itemize}
\end{frame}



\begin{frame}{Probabilistic choice}
With the $\varepsilon$'s unobserved, we must consider choices as probabilistic instead of certain. The Probability that $i$ chooses alternative $j$ is:
\begin{align}
P_{ij}&=\Pr(u_{ij}+\varepsilon_{ij}>u_{ij'}+\varepsilon_{ij'}  \,\,\forall\,\,  j'\neq j)\\
&=\Pr(\varepsilon_{ij'}-\varepsilon_{ij}<u_{ij}-u_{ij'}  \,\,\forall\,\,  j'\neq j)\\
&=\int_{\varepsilon}I(\varepsilon_{ij'}-\varepsilon_{ij}<u_{ij}-u_{ij'}  \,\,\forall\,\,  j'\neq j)f(\varepsilon)d\varepsilon
\end{align}
\end{frame}



\begin{frame}{Transformations of utility}
Note that, regardless of what distributional assumptions are made on the $\varepsilon$'s, the probability of choosing a particular option does not change when we:
\begin{enumerate}
\item Add a constant to the utility of all options (utility is relative to one of the options, only differences in utility matter)
\item Multiply by a positive number (need to scale something, generally the variance of the $\varepsilon$'s)
\end{enumerate}
This is just like in consumer choice theory: utility is ordinal, and so is invariant to the above two transformations
\end{frame}



\begin{frame}{Variables}
Suppose we have:
\begin{align*}
u_{i1}&=\alpha Male_i+\beta_1 X_i + \gamma Z_1\\
u_{i2}&=\alpha Male_i+\beta_2 X_i+\gamma Z_2
\end{align*}
Since only differences in utility matter:
\begin{equation*}
u_{i1}-u_{i2}=(\beta_1-\beta_2)X_i+\gamma (Z_1-Z_2)
\end{equation*}
\begin{itemize}
\item Thus, we cannot tell whether men are happier than women, but can tell whether men have a preference for a particular option over another. 
\item We can only obtain \textbf{differenced} coefficient estimates on $X$'s, and can obtain an estimate of a coefficient that is constant across choices only if the variable it is multiplying varies by choice.
\end{itemize}
\end{frame}



\begin{frame}{Number of error terms}
Similar to socio-demographic characteristics, there are restrictions on the number of error terms.
\bigskip
Recall that he probability $i$ will choose $j$ is given by:
\begin{eqnarray*}
P_{ij}&=&\Pr(u_{ij}+\varepsilon_{ij}>u_{ij'}+\varepsilon_{ij'}  \,\,\forall\,\,   j'\neq j)\\
&=&\Pr(\varepsilon_{ij'}-\varepsilon_{ij}<u_{ij}-u_{ij'}  \,\,\forall\,\,   j'\neq j)\\
&=&\int_{\varepsilon}I(\varepsilon_{ij'}-\varepsilon_{ij}<u_{ij}-u_{ij'}  \,\,\forall\,\,   j'\neq j)f(\varepsilon)d\varepsilon
\end{eqnarray*}
where the integral is $J$-dimensional.
\end{frame}



\begin{frame}{Number of error terms (cont'd)}
But we can rewrite the last line as $J-1$ dimensional integral over the differenced $\varepsilon$'s:
\begin{displaymath}
P_{ij}=\int_{\tilde{\varepsilon}}I(\tilde{\varepsilon}_{ij'}<\tilde{u}_{ij'}  \,\,\forall\,\,   j'\neq j)g(\tilde{\varepsilon})d\tilde{\varepsilon}
\end{displaymath}
Note that this means one dimension of $f(\varepsilon)$ is not identified and must therefore be normalized.
\end{frame}



\begin{frame}{Derivation of Logit Probability}
Consider the case when the choice set is $\{1,2\}$.  The Type 1 extreme value cdf for $\varepsilon_2$ is:
\begin{displaymath}
F(\varepsilon_2)=e^{-e^{(-\varepsilon_2)}}
\end{displaymath}
To get the probability of choosing $1$, substitute in for $\varepsilon_2$ with $\varepsilon_1+u_1-u_2$:
\begin{displaymath}
Pr(d_1=1|\varepsilon_1)=e^{-e^{-(\varepsilon_1+u_1-u_2)}}
\end{displaymath}
But $\varepsilon_1$ is unobserved so we need to integrate it out (see Appendix to these slides if you want the math steps)
\end{frame}



\begin{frame}{Derivation of Logit Probability}
In the end, we can show that, for any model where there are two choice alternatives and $\varepsilon$ is drawn from the Type 1 extreme value distribution,

\begin{displaymath}
P_{i1}=\frac{\exp(u_{i1}-u_{i2})}{1+\exp(u_{i1}-u_{i2})},P_{i2}=\frac{1}{1+\exp(u_{i1}-u_{i2})}
\end{displaymath}

Suppose we have a data set with $N$ observations. The log likelihood function we maximize is then:
\begin{displaymath}
\ell(\beta,\gamma)=\sum_{i=1}^N(d_{i1}=1)(u_{i1}-u_{i2})-\ln\left(1+\exp(u_{i1}-u_{i2})\right)
\end{displaymath}

\end{frame}



\begin{frame}{Derivation of Probit Probability}
In the probit model, we assume that $\varepsilon$ is Normally distributed. So for a binary choice we have:

\begin{displaymath}
P_{i1}=\Phi\left(u_{i1}-u_{i2}\right),P_{i2}=1-\Phi\left(u_{i1}-u_{i2}\right)
\end{displaymath}
where $\Phi\left(\cdot\right)$ is the standard normal cdf

The log likelihood function we maximize is then:
\begin{displaymath}
\ell(\beta,\gamma)=\sum_{i=1}^N(d_{i1}=1)\ln\left(\Phi\left(u_{i1}-u_{i2}\right)\right)+(d_{i2}=1)\ln\left(1-\Phi\left(u_{i1}-u_{i2}\right)\right)
\end{displaymath}

\end{frame}



\begin{frame}{Pros \& Cons of Logit \& Probit}
Logit model:
\begin{itemize}
\item Has a much simpler objective function 
\item Is by far most popular
\item ... but has more restrictive assumptions about how people substitute choices
\item (this is known as the Independence of Irrelevant Alternatives or IIA assumption)
\end{itemize}

Probit model:
\begin{itemize}
\item Much more difficult to estimate 
\item ... but can accommodate more realistic choice patterns
\end{itemize}

\end{frame}



\section{Estimation}
\begin{frame}[fragile]{Estimation in \texttt{R}}
The R function \texttt{glm} is the easiest way to estimate a binomial logit or probit model:

\begin{lstlisting}[language=R]
library(mlogit)
data(Heating) # load data on residential heating choice in CA
levels(Heating$depvar) <- c("gas","gas","elec","elec","elec")
estim <- glm(depvar ~ income+agehed+rooms+region,
             family=binomial(link='logit'),data=Heating))
print(summary(estim))
\end{lstlisting}

\end{frame}




\begin{frame}[fragile]{Interpreting the coefficients}
Estimated coefficients using the code in the previous slide:

\begin{lstlisting}[language=R]
Coefficients:
              Estimate Std. Error z value Pr(>|z|)
(Intercept)  -0.599142   0.252032  -2.377   0.0174 *
income        0.015579   0.027905   0.558   0.5766
agehed       -0.006535   0.003342  -1.955   0.0505 .
rooms         0.024291   0.026916   0.902   0.3668
regionscostl -0.053096   0.126665  -0.419   0.6751
regionmountn  0.041827   0.169787   0.246   0.8054
regionncostl -0.219136   0.137692  -1.591   0.1115
---
Signif. codes:  0 '***' 0.001 '**' 0.01 '*' 0.05 '.' 0.1 ' ' 1
\end{lstlisting}

\end{frame}



\begin{frame}{Interpreting the coefficients}
\begin{itemize}
\item Positive coefficients $\Rightarrow$ household more likely to choose the non-baseline alternative (in this case: electric)
    \begin{itemize}
    \item Whatever the first \texttt{level} of the factor dependent variable is will be the "baseline" alternative
    \end{itemize}
\item Negative coefficients imply the reverse
\item Coefficients \textbf{not} linked to changes in probability of choosing the alternative (since probability is a nonlinear function of $X$)
\end{itemize}
\end{frame}



\begin{frame}[fragile]{Forming predictions}
To get predicted probabilities for each observation in the data:

\begin{lstlisting}[language=R]
Heating$predLogit <- predict(estim, newdata = Heating, type = "response")
print(summary(Heating$predLogit))
\end{lstlisting}

\end{frame}





\begin{frame}[fragile]{Estimating a probit model}
For the probit model, we repeat the same code, except change the ``link'' function from ``logit'' to ``probit''

\begin{lstlisting}[language=R]
estim2 <- glm(depvar ~ income+agehed+rooms+region,
             family=binomial(link='probit'),data=Heating))
print(summary(estim2))
Heating$predProbit <- predict(estim2, newdata = Heating, type = "response")
print(summary(Heating$predProbit))
\end{lstlisting}

\end{frame}



\begin{frame}{A simple counterfactual simulation}
\begin{itemize}
\item We talked a lot about doing counterfactual comparisons, but how do we \emph{actually} do it?
\item Let's show how to do this on a previous example. Suppose that we introduce a policy that makes richer people more likely to use electric heating.
\item Mathematically, what does this look like?
\item It would correspond to an increase in the parameter in front of \emph{income} in our regression
\end{itemize}
\end{frame}



\begin{frame}[fragile]{A simple counterfactual simulation}
\begin{itemize}
\item Suppose the coefficient increased by a factor of 4. What is the new share of gas vs. electricity usage?
\end{itemize}

\begin{lstlisting}[language=R]
estim$coefficients["income"] <- 4*estim$coefficients["income"]
Heating$predLogitCfl <- predict(estim, newdata = Heating, type = "response")
print(summary(Heating$predLogitCfl))
\end{lstlisting}

This policy would increase electric usage by 7 percentage points (from 22\% to 29\%)

\end{frame}




\section{Sample selection}
\begin{frame}[fragile]{Discrete choice models and sample selection bias}
\begin{itemize}
\item Discrete choice models are common tools used to evaluate sample selection bias
\item Why? Because variables that are MNAR can be thought of as following a utility-maximizing process
\item Examples:
    \begin{itemize}
    \item Suppose you want to know what the returns to schooling are, but you only observe wages for those who currently hold jobs
    \item As a result, your estimate of the returns to schooling might be invalidated by the non-randomness of the sample of people who are currently working
    \item How to get around this? Use a discrete choice model (This was the problem we ran into in PS7, if you recall)
    \end{itemize}
\end{itemize}
\end{frame}


\begin{frame}{Heckman selection correction}
The Heckman selection model specifies two equations:
\begin{align*}
    u_{i}&= \beta x_{i} + \nu_{i} \\
    y_{i} &= \gamma z_{i} + \varepsilon_{i}
\end{align*}
\begin{itemize}
    \item The first equation is a utility maximization problem, determining if the person is in the labor force. Can think of $\nu_{i}$ as ``desire to work''
    \item $x_{i}$ may include: number of children in the household
    \item The second equation is the log wage equation, where $y_i$ is only observed for people who are in the labor force.
    \item To solve the model, one needs to use the so-called ``Heckit'' model, which involves adding a correction term in the wage equation which accounts for the fact that workers are not randomly selected.
\end{itemize}
\end{frame}




\begin{frame}[fragile]{Estimating Heckman selection in \texttt{R}}
    \texttt{R} has a package called \texttt{sampleSelection} which incorporates the Heckman selection model\footnote{This code taken from Garrett Glasgow's website: \url{http://www.polsci.ucsb.edu/faculty/glasgow/ps207/ps207_class6.r}}

\begin{lstlisting}[language=R]
library(sampleSelection)
data('Mroz87')
Mroz87$kids <- (Mroz87$kids5 + Mroz87$kids618) > 0
# Comparison of linear regression and selection model
outcome1 <- lm(wage ~ exper, data = Mroz87)
summary(outcome1)
selection1 <- selection(selection = lfp ~ age + I(age^2) + faminc + kids + educ,
outcome = wage ~ exper, data = Mroz87, method = '2step')
summary(selection1)
\end{lstlisting}

\end{frame}



\begin{frame}[fragile]{Estimation output}
    Output from a regression of wage on experience:

\begin{lstlisting}[language=R]
Coefficients:
            Estimate Std. Error t value Pr(>|t|)
(Intercept)  1.30434    0.18937   6.888 1.20e-11 ***
exper        0.10067    0.01419   7.093 3.03e-12 ***
---
\end{lstlisting}

\end{frame}


\begin{frame}[fragile]{Estimation output}
    Output from the Heckman selection model: (edited for length)

\begin{lstlisting}[language=R]
Probit selection equation:
               Estimate Std. Error t value Pr(>|t|)
(Intercept) -4.157e+00  1.402e+00  -2.965 0.003126 **
kidsTRUE    -4.490e-01  1.309e-01  -3.430 0.000638 ***
Outcome equation:
               Estimate Std. Error t value Pr(>|t|)
(Intercept)  7.12492    0.80425   8.859   <2e-16 ***
exper        0.02962    0.02059   1.439    0.151
Multiple R-Squared:0.0823,      Adjusted R-Squared:0.0779
Error terms:
                Estimate Std. Error t value Pr(>|t|)
invMillsRatio   -5.075      1.108  -4.581 5.42e-06 ***
sigma            4.977         NA      NA       NA
rho             -1.020         NA      NA       NA
\end{lstlisting}

\end{frame}



\begin{frame}{Reading the output}
\begin{itemize}
    \item Because there are two equations, there are now more parameters
    \item Using just the regression on workers led us to believe the returns to experience were $\approx 10\%$
    \item Taking into account the selectivity of labor force participants leads us to conclude the returns to experience are much lower ($\approx 3\%$ and not statistically different from zero)
    \item Viability of the model depends on the assumption that's made: in this case, that having children only affects labor supply preferences and doesn't affect wages
    \begin{itemize}
        \item Wage discrimination against mothers in the labor market would invalidate this assumption
        \item Back to the idea that to get causal inference we have to impose more assumptions
    \end{itemize}
\end{itemize}
\end{frame}



\section{Dynamic Discrete Choice}
\begin{frame}{The optimal stopping problem}
\begin{itemize}
\item Much of life is concerned with knowing when to stop:
    \begin{itemize}
    \item How many people to date before making/accepting a marriage proposal
    \item How much to study for upcoming exams
    \item How long to ``hodl'' an asset
    \end{itemize}
\item All of the above cases involve forming expectations about:
    \begin{enumerate}
    \item The long-run value of making a particular choice
    \item ... relative to the long-run value of alternatives
    \end{enumerate}
\item Expectations about the future imply that we need to think ``dynamically'' (i.e. think over the longterm)
\item Today we'll go through the math on how to do this
\end{itemize}
\end{frame}



\begin{frame}{Relation to reinforcement learning}
\begin{itemize}
\item Reinforcement learning is based on the optimal stopping problem
\item At each state $X$ (e.g. game board configuration), observe reward $y$ (e.g. win probability)
\item In each period (i.e. gameplay turn), choose the decision that maximizes the (present value) expected reward
\item With structural models, ``reward'' is utility
\end{itemize}
\end{frame}



\begin{frame}{Dynamic discrete choice models}
With \emph{dynamic} models, need a way to quantify present value of utility

Individual $i$'s \textbf{flow utility} for option $j$ at time $t$ is:
\begin{eqnarray*}
    U_{ijt}&=&u_{ijt}+\varepsilon_{ijt}\\
           &=&X_{it}\alpha_j+\varepsilon_{ijt}
\end{eqnarray*}
Individual chooses $d_{it}$ to maximize \textbf{expected lifetime utility}
\begin{displaymath}
    \max_{d_{it}} V = E\left\{\sum_{\tau=t}^T\sum_j\beta^{\tau-t}(d_{it}=j)U_{ijt}\right\}
\end{displaymath} 
\begin{itemize}
    \item $V$ is the \emph{value function}
    \item $\beta\in\left(0,1\right)$ is the \emph{discount factor}
    \item $T$ is the \emph{time horizon}
\end{itemize}
\end{frame}




\begin{frame}{Expectations}
\begin{itemize}
\item Expectations taken over future states ($X$'s) \textbf{and} errors
\item $\varepsilon$'s are iid over time
\item Future states are not affected by $\varepsilon$'s except through current and past choices:
    \begin{displaymath}
        E\left(X_{t+1}|d_t,...,d_1,\varepsilon_t,...,\varepsilon_{1}\right)=E\left(X_{t+1}|d_t,...,d_1\right)
    \end{displaymath}
\end{itemize}
\end{frame}




\begin{frame}{Human behavior vs. reinforcement learning}
\begin{itemize}
\item In reinforcement learning, we typically don't have $\varepsilon$, unless we want to allow for ``curiosity''
\item Transitions in $X$ much more dominant factor (e.g. if I move here, opponent will move there, ...)
\item Real-life example of uncertainty in $\varepsilon$'s:
    \begin{itemize}
        \item ``My significant other might take a job in another city next year, so if I want to move with him/her, I may not want to take this job offer today.''
    \end{itemize}
\item Real-life example of uncertainty in $X$'s:
    \begin{itemize}
        \item ``I might get laid off next year, which will influence my ability to pay off my car loan, so I might want not want to buy this Mercedes today, since my (expected) permanent income might be lower than my current income.''
    \end{itemize}
\end{itemize}
\end{frame}




\begin{frame}{Dynamic programming \& the Bellman equation}
\begin{itemize}
\item We want to maximize the value function $V$
\item It's helpful to write the value function as a recursive expression, where we separate out today's decision from all future decisions (this is called the \emph{Bellman equation}, or the \emph{dynamic programming problem})
\item The payoff from choosing alternative $j$ today is the \emph{flow utility} $= u_{ijt}$ from earlier in these slides
\item The payoff from choosing alternative $j$ in the future is the expected future utility conditional on choosing $j$ today
\end{itemize}
How do we solve the Bellman equation?
\begin{itemize}
\item Requires solving backwards, just like in a dynamic game (cf. subgame perfect Nash equilibrium)
\end{itemize}
\end{frame}




\begin{frame}{Two Period Example}
    Consider the utility of choice $j$ in the last period:
    \begin{align*}
        U_{ijT}&=u_{ijT}+\varepsilon_{ijT}\\
               &=X_{iT}\alpha_j+\varepsilon_{ijT}
    \end{align*}
    Define the \textbf{conditional valuation function} for choice $j$ as the flow utility of $j$ minus the associated $\varepsilon$ plus the expected value of future utility conditional on $j$:
    \begin{displaymath}
        v_{ijT-1}=u_{ijT-1}+\beta E\max_{k\in J}\left\{u_{ikT}+\varepsilon_{ikT}|d_{iT-1}=j\right\}
    \end{displaymath} 
    where $\beta$ is the discount factor.

    \bigskip
    Suppose $X_{iT}$ was deterministic given $X_{iT-1}$ and $d_{T-1}$ and the $\varepsilon$'s are Type 1 extreme value. What would the $E\max$ expression be? $\left[\ln\sum_{k}\exp\left(u_{ikT}\right)\right]$
\end{frame}




\begin{frame}{Two Period Example (cont'd)}
    For $J=2$ the log likelihood would then look like:

    \begin{displaymath}
        L(\alpha)=\sum_{i=1}^N\sum_{t=1}^T(d_{i1t}=1)(v_{i1t}-v_{i2t})-\ln\left(1+\exp(v_{i1t}-v_{i2t})\right)
    \end{displaymath}

    where 

    \begin{displaymath}
        v_{ijt}=u_{ijt}+\beta E\max_{k\in J}\left\{v_{ikt+1}+\varepsilon_{ikt+1}|d_{it}=j\right\}
    \end{displaymath}

    and where

    \begin{displaymath}
        u_{ijt}=X_{it}\alpha_{j}
    \end{displaymath}

    Note: if $T=2$ then $v_{ikt+1} = u_{ikT}$

\end{frame}




\begin{frame}{Estimating a dynamic discrete choice model in \texttt{R}}
\begin{itemize}
    \item Because we have to loop backwards through time, we can't use a canned function like \texttt{lm()}
    \item Requires us to write a custom likelihood function
    \item This is because the flow utility parameters ($\alpha_j$) appear in the flow utility function in \emph{each} period
        \begin{itemize}
            \item Side note: We don't typically estimate the discount factor ($\beta$) but instead assume a fixed value (most common: 0.90 or 0.95)
        \end{itemize}
    \item To do this, write down an objective function (i.e. log likelihood function) and use \texttt{nloptr} to estimate the $\alpha$'s
    \item Once you have the $\alpha$'s you can do counterfactual simulations
    \item These simulations are likely to be more realistic because the model has incorporated forward-looking behavior
\end{itemize}
\end{frame}


\begin{frame}[fragile]{Objective function}
\begin{lstlisting}[language=R]
objfun <- function(alpha,Choice,age) {
    J <- 2
    a  <- alpha[3]*(1-diag(J))
    
    u1 <- matrix(0, N, T)
    u2 <- matrix(0, N, T)
    for (t in 1:T) {
        u1[ ,t] <- 0*age[ ,t]
        u2[ ,t] <- alpha[1] + alpha[2]*age[ ,t]
    }

\end{lstlisting}
(continued on next slide)
\end{frame}


\begin{frame}[fragile,plain]
\begin{lstlisting}[language=R]
    Like <- 0
    for (t in T:1) {
        for (j in 1:J) {
            # Generate FV
            dem <- exp(u1[ ,t] + a[1,j]+fv[ ,1,t+1])+
                   exp(u2[ ,t] + a[2,j]+fv[ ,2,t+1])
            fv[ ,j,t] <- beta*(log(dem)-digamma(1))
            p1 <- exp(u1[ ,t] + a[1,j] + fv[ ,1,t+1])/dem
            p2 <- exp(u2[ ,t] + a[2,j] + fv[ ,2,t+1])/dem
            Like <- Like - (LY[ ,t]==j)*((Choice[ ,t]==1)*log(p1)+(Choice[ ,2]==2)*log(p2))
        }
    }
    return ( sum(Like) )
}
\end{lstlisting}
\end{frame}    



\begin{frame}[fragile]{Calling \texttt{nloptr}}
\begin{lstlisting}[language=R]
## initial values
theta0 <- runif(3) #start at uniform random numbers equal to number of coefficients

## Algorithm parameters
options <- list("algorithm"="NLOPT_LN_NELDERMEAD","xtol_rel"=1.0e-6,"maxeval"=1e4)

## Optimize!
result <- nloptr( x0=theta0,eval_f=objfun,opts=options,Choice=Choice,age=age)
print(result)
\end{lstlisting}
\end{frame}    



\appendix

\section*{Bonus slides}

\begin{frame}[noframenumbering]{Derivation of Logit Probability}
\begin{eqnarray*}
Pr(d_1=1)&=&\int_{-\infty}^{\infty}\left(e^{-e^{-(\varepsilon_1+u_1-u_2)}}\right)f(\varepsilon_1)d\varepsilon_1\\
&=&\int_{-\infty}^{\infty}\left(e^{-e^{-(\varepsilon_1+u_1-u_2)}}\right)e^{-\varepsilon_1}e^{-e^{-\varepsilon_1}}d\varepsilon_1\\
&=&\int_{-\infty}^{\infty}\exp\left(-e^{-\varepsilon_1}-e^{-(\varepsilon_1+u_1-u_2)}\right)e^{-\varepsilon_1}d\varepsilon_1\\
&=&\int_{-\infty}^{\infty}\exp\left(-e^{-\varepsilon_1}\left[1+e^{u_2-u_1}\right]\right)e^{-\varepsilon_1}d\varepsilon_1
\end{eqnarray*}
\end{frame}

\begin{frame}[noframenumbering]{Derivation of Logit Probability}
Now need to do the substitution rule where $t=\exp(-\varepsilon_1)$ and $dt=-\exp(-\varepsilon_1)d\varepsilon_1$.  

\bigskip

Note that we need to do the same transformation of the bounds as we do to $\varepsilon_1$ to get $t$.  Namely, $\exp(-\infty)=0$ and $\exp(\infty)=\infty$.
\end{frame}

\begin{frame}[noframenumbering]{Derivation of Logit Probability}
Substituting in then yields:
\begin{eqnarray*}
Pr(d_1=1)&=&\int_{\infty}^0\exp\left(-t\left[1+e^{(u_2-u_1)}\right]\right)(-dt)\\
&=&\int_0^{\infty}\exp\left(-t\left[1+e^{(u_2-u_1)}\right]\right)dt\\
&=&\begin{array}{c|}\frac{\exp\left(-t\left[1+e^{(u_2-u_1)}\right]\right)}{-\left[1+e^{(u_2-u_1)}\right]}\end{array}^{\infty}_{0}\\
&=&0-\frac{1}{-\left[1+e^{(u_2-u_1)}\right]}=\frac{\exp(u_1)}{\exp(u_1)+\exp(u_2)}
\end{eqnarray*}
\end{frame}

%\begin{frame}[noframenumbering]{References}
%\bibliographystyle{jpe}
%\bibliography{biblio}
%\end{frame}

\end{document}

